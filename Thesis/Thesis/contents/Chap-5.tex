\chapter{To-Do List}
\label{chp:TD}

%%%%%%%%%%%%%%%%%%%%%%%%%%%%%%%%%%%%%%%%%%%%%%%%%%%%%%%%%%%%%%%%%%%%%%%
\section{Document}

\subsection{Introduction}

Define aim and objectives properly

-implement generative design process to construct basic elements

-implement generative design process to construct soft bodies from basic elements

-use generative design process to design soft actuator meeting some goal

-compare results to previous work



Define scope and assumptions properly

-hyper-elastic non-linear materials being inflated

-two dimensions

\subsection{Literature Review}

-define L-systems appropriately

-discuss and define CPPNs

-add illustrative diagrams where necessary

\subsection{Material Testing}

List other materials and given properties if applicable

-Smooth Sil 950

-Ecoflex 0030



Describe testing process in detail

-specimen preparation

	--wear nitrile gloves
	
	--sanitise workspace
	
	--mix materials in 1:1 ratio
	
	--mix until no streaks
	
	--degas
	
	--pour into tensile specimen and compression specimen mould
	
	--degas
	
	--even out surface
	
	--leave to set for 4 hours
	
-specimen testing

	--describe ISO standards appropriately
	
	--use Instron machine
	
	--100 kN load cell
	
	--clamp grips vs roller grips
	
	--long travel extensometer vs DIC
	

\subsection{Software}

Discuss coding approach in more detail

Discuss analysis of results in more detail

Refine layout and diagram quality

Eliminate unnecessary diagrams/translate to writing

\chapter{Conclusions}
\label{chp:C}

%%%%%%%%%%%%%%%%%%%%%%%%%%%%%%%%%%%%%%%%%%%%%%%%%%%%%%%%%%%%%%%%%%%%%%%

\section{Summary}

This project aimed to implement a scale-invariant generative design process capable of suggesting replicable soft robotic actuator components based on methods discussed in a review of existing literature.

A generative design pipeline was constructed. Soft robotic components were designed by software en masse according to specified unit generation methods. Unit generation methods implemented pattern generation approaches with elements of randomness. Units were scored against objective functions and well-performing units could be isolated and inspected manually.

The design pipeline was designed with scalability and modularity in mind. Unit generation methods are scalable to differing resolutions. Compacting units into grids of varying sizes replicates behaviour of units at the individual scale. The pipeline is easily modifiable. New objective functions, unit generation methods, and analysis approaches are easily implementable. The pipeline is accessible online to anyone.

Physical models replicating selected unit designs were manufactured. Physical model deformations were compared to predicted deformations and found to be accurate.

The methodology employed in this project proved to be useful and effective as a template for investigations of soft robotic design spaces. Its modularity allows for future research to easily make use of it.

\section{Future Work}

There is potential for improvements to be made to the software pipeline as well as the physical model validation. Evolutionary algorithms may be tested more extensively with alternate parameters. Larger simulations may be run on more powerful machines in order to explore even more possible solutions. Higher quality moulds may be manufactured to increase the accuracy of the physical models.
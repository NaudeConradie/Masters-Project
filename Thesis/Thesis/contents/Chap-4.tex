\chapter{Results}
\label{chp:R}

%%%%%%%%%%%%%%%%%%%%%%%%%%%%%%%%%%%%%%%%%%%%%%%%%%%%%%%%%%%%%%%%%%%%%%%

This chapter illustrates results obtained from sample simulations of large populations of units according to the outlined unit generation methods in Section~\ref{ssec:uapg} and the analysis approaches outlined in Section~\ref{ssec:aa}. The software pipeline described in Section~\ref{sec:SW} was used to generate the units and obtain the results.

Table~\ref{tab:defpar} contains the default parameter values. Unless specified otherwise, parameters were set to the values in Table~\ref{tab:defpar}. Additional parameters for evolutionary algorithms are included in Table~\ref{tab:eaeventpar}.

% Please add the following required packages to your document preamble:
% \usepackage{booktabs}
\begin{table}[H]
\centering
\caption{Default simulation parameter values}
\label{tab:defpar}
\begin{tabular}{@{}lc@{}}
\toprule
\multicolumn{1}{c}{\textbf{Parameter}} & \textbf{Value}                 \\ \midrule
\textit{n\_u}                          & 1000                           \\
\textit{y\_e}                          & 15                             \\
\textit{e\_s}                          & 10                             \\
\textit{b}                             & 3                              \\
\textit{n\_steps}                      & 5                              \\
\textit{d\_mag}                        & $\frac{y\_e\times e\_s}{2}=75$ \\
\textit{p\_mag}                        & 0.025                          \\ \midrule
\multicolumn{2}{c}{\textbf{Monte Carlo Analysis}}                       \\ \midrule
\textit{n\_u}                          & 1000                           \\ \midrule
\multicolumn{2}{c}{\textbf{Evolutionary Algorithm}}                     \\ \midrule
\textit{n\_u}                          & 50                             \\
\textit{gen}                           & 50                             \\ \bottomrule
\end{tabular}
\end{table}

% Please add the following required packages to your document preamble:
% \usepackage{booktabs}
\begin{table}[H]
\centering
\caption{Evolutionary algorithm event parameter values}
\label{tab:eaeventpar}
\begin{tabular}{@{}lcc@{}}
\toprule
\multicolumn{1}{c}{\textbf{Event}} & \textbf{Probability (\%)} & \textbf{\begin{tabular}[c]{@{}c@{}}Potential Number\\ of Occurences\end{tabular}} \\ \midrule
Crossover       & 50 & 1 \\
Random mutation & 10 & 2 \\
Biased mutation & 50 & 2 \\ \bottomrule
\end{tabular}
\end{table}

\section{Random Unit Generation}

\subsection{Monte Carlo Analysis}

Parameter ranges are outlined in Table~\ref{tab:ranmc}.

% Please add the following required packages to your document preamble:
% \usepackage{booktabs}
\begin{table}[H]
\centering
\caption{Random unit generation parameters for a Monte Carlo analysis}
\label{tab:ranmc}
\begin{tabular}{@{}lcc@{}}
\toprule
\multicolumn{1}{c}{\textbf{Parameter}} & \textbf{Minimum} & \textbf{Maximum} \\ \midrule
Seed                                   & 1                & 1000             \\
Number of elements removed             & 0                & 81               \\ \bottomrule
\end{tabular}
\end{table}

Figures~ and ~ represent the relationships between the parameters in Table~\ref{tab:ranmc} and the score as calculated in Section~.

\section{L-System Unit Generation}

\subsection{Monte Carlo Analysis}

Parameter ranges are outlined in Table~\ref{tab:lsmc}.

% Please add the following required packages to your document preamble:
% \usepackage{booktabs}
\begin{table}[H]
\centering
\caption{L-System unit generation parameters for a Monte Carlo analysis}
\label{tab:lsmc}
\begin{tabular}{@{}lcc@{}}
\toprule
\multicolumn{1}{c}{\textbf{Parameter}} & \textbf{Minimum} & \textbf{Maximum} \\ \midrule
Seed                                   & 1                & 1000             \\
Axiom ID                               & 1                & 12               \\
Number of rules                        & 1                & 4                \\
Rule length                            & 2                & 5                \\
Number of iterations                   & 1                & 5                \\ \bottomrule
\end{tabular}
\end{table}

Figures~ to ~ represent the relationships between the parameters in Table~\ref{tab:lsmc} and the score as calculated in Section~.

\section{CPPN Unit Generation}

\subsection{Monte Carlo Analysis}

Parameter ranges are outlined in Table~\ref{tab:cppnmc}.

% Please add the following required packages to your document preamble:
% \usepackage{booktabs}
\begin{table}[H]
\centering
\caption{CPPN unit generation parameters for a Monte Carlo analysis}
\label{tab:cppnmc}
\begin{tabular}{@{}lcc@{}}
\toprule
\multicolumn{1}{c}{\textbf{Parameter}} & \textbf{Minimum} & \textbf{Maximum} \\ \midrule
Seed                                   & 1                & 1000             \\
Model ID                               & 1                & N/A              \\
Scale                                  & 1                & N/A              \\
Number of hidden layers                & 2                & 10               \\
Size of the initial hidden layer       & 2                & 32               \\
Element removal threshold              & 0                & 100              \\ \bottomrule
\end{tabular}
\end{table}

Figures~ to ~ represent the relationships between the parameters in Table~\ref{tab:cppnmc} and the score as calculated in Section~.
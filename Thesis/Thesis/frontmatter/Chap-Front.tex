\begin{abstract}[english]%===================================================

Designing soft robotic components is a novel problem that traditional design methods may not be equipped to solve. Generative design approaches in conjunction with accurate FEM modelling is presented as a potential method of obtaining interesting and unique soft robotic actuator designs.

A review is done of soft robotics, computationally modelling non-linear materials, generative design methods, and applicable mathematical concepts. Based on the review, a methodology is developed for the design of soft robotic actuators.

A software pipeline is constructed for the exploration of a 2D design space. Pattern generating methods including Lindenmayer systems and CPPNs are used to formulate internal geometries of actuator components.  Generative design approaches including Monte Carlo simulations and evolutionary algorithms are used to generate and evaluate large populations of actuator units. FEM is used to accurately model and simulate unit properties and behaviours. Objective functions are defined to evaluate unit performances.

Results are found to indicate the validity of the design methodology. Physical models representative of successful simulated models are cast with a modular mould. These models are tested and compared to simulated results. FEM models were found to be accurate predictors of the shape deformation of actuator units.

\end{abstract}


\begin{abstract}[afrikaans]%=================================================

Die ontwerp van sagte robotkomponente is 'n nuwe probleem wat nie noodwendig opgelos kan word deur tradisionele ontwerpmetodes nie. Generatiewe ontwerpbenaderings tesame met akkurate FEM-modellering word voorgestel as 'n moontlike metode om interessante en unieke sagte robotaktuatorontwerpe te bekom.

Navorsing word gedoen oor sagte robotika, die virtuele modellering van nie-lineêre materiale, generatiewe ontwerpmetodes en toepaslike wiskundige konsepte. Op grond van die navorsing word 'n metodologie ontwikkel vir die ontwerp van sagte robotaktuators.

'n Sagtewarepypleiding word ontwerp vir die verkenning van 'n 2D ontwerpruimte. Patroongenererende metodes, insluitend Lindenmayer-stelsels en CPPN's, word gebruik om interne dimensies van aktuators te formuleer. Generatiewe ontwerpbenaderings, insluitend Monte Carlo-simulasies en evolusionêre algoritmes, word gebruik om groot populasies van aktuators te genereer en te evalueer. FEM word gebruik om aktuator eienskappe en gedrag akkuraat te modelleer en te simuleer. Doelwitfunksies word gedefinieer om aktuator gedrag te evalueer.

Daar word gevind dat resultate die geldigheid van die ontwerpmetodologie aandui. Fisiese modelle wat verteenwoordigend is van suksesvolle gesimuleerde modelle word in 'n modulêre bak gegiet. Hierdie modelle word getoets en vergelyk met die gesimuleerde resultate. Daar is gevind dat FEM-modelle akkurate voorspellers is vir die vervorming van aktuators.

\end{abstract}


\chapter{Acknowledgements}%==================================================

I would like to express my sincere gratitude to the following people and organisations:

My wife, Anmari Conradie, for her unending love, support and patience throughout this project

My parents and in-laws for their love and assistance

DebMarine Namibia, for sponsoring my studies



%\chapter{Dedications}%=======================================================
 %\vfill
 %\begin{center}\itshape
 	%I would like to dedicate this thesis to my wife, Anmari Conradie, for her support and understanding throughout this project.
 %\end{center}
 %\vfill
 %\clearpage

%============================================================================
\endinput

\chapter{Literature Review}
\label{chp:LR}


%%%%%%%%%%%%%%%%%%%%%%%%%%%%%%%%%%%%%%%%%%%%%%%%%%%%%%%%%%%%%%%%%%%%%%%
\section{Soft Robotics}

The field of soft robotics is relatively new and developmental. Soft robots inherently differ from traditional robots. They are constructed from compliant and pliable materials. Soft robots generally have fixed joints and locomotive actuators between joints, as opposed to traditional robots which usually have locomotive actuated joints connected by rigid sections. \cite{Whitesides2018}

Soft robots offer many advantages over traditional robots. They provide a safer working environment around humans and with fragile materials, as they are much lighter, more pliable and move slower and with less force than traditional robots. They require lower tolerances for manufacture due to their inherent less precise nature. The fact that they are very lightweight relative to traditional robots may also offer many advantages. \cite{Whitesides2018}

\subsection{Actuators}

Actuators are components that cause controlled motion, generally used in robotics and machinery \cite{Sekhar2012}. There are currently a few major types of soft robotic actuators in use \cite{Boyraz2018}.

\subsubsection{Actuator Types}

Shape Memory Alloys (SMA) are metallic alloys capable of being formed into a specific shape while above an inherent transformation temperature, as well as being formed into another shape below the transformation temperature. When the material is then heated or cooled above or below the transformation temperature, it reforms into those respective shapes. This property exists due to the transition between the martensite phase of the material below the transformation temperature, and the austenite phase above the transformation temperature. SMA actuators are heated by applying a current directly to the material. SMA actuators are small, lightweight, silent, and have a high force-to-weight ratio. When shaped straight, they can exert high forces, but only achieve small displacements relative to their length. When coiled, they can extend more, but exert smaller forces. \cite{Villoslada2015}

\hl{insert illustrative figure}

Shape Memory Polymers (SMP) are similar to SMAs, consisting of smart polymers with the same shape memory properties, instead of metallic alloys. The initial is shape is determined during the manufacturing process. The transformed shape is obtained by cooling the SMP and shaping it as desired. SMPs use electricity or light as a heat source for transformation \cite{Behl2207}. They have a high deformation capacity and shape recovery. They are lighter, cheaper and easier to produce than SMAs. They are limited in size due to their low recovery stresses {Rodriguez2016}.

Dieelectric/Electrically-Actuated Polymers (DEAP) consist of layers of polymers interspersed with conductive material. When the conductive material receives an electrical input, a chemical reaction occurs that causes a change in volume across the layers. This causes the layers to bend in a predetermined direction. DEAPs are lightweight, silent and use little energy. They are biocompatible and functional in water. They are well-suited to mimicking real muscles. Their reactions under high voltages are not fully understood yet and accurately modelling them is highly complicated. \cite{Mutlu2014}

Electro-Magnetic Actuators (EMA) make use of magnetic microparticles within a polymer matrix. The particles are manipulated to cause motion by an external magnetic field from an electromagnet. This allows for a wide range of motion by varying the orientation and magnitude of the electromagnetic field. EMAs are small, require low voltages and are efficient. They have quick response times and high dynamic ranges. They are still an emerging technology in the early stages of development. \cite{Do2018}

Fluid Elastomeric Actuators (FEA) use soft polymeric structures with internal geometry designed for specific types of motion when driven by fluid pressure. Fluid pressure may be obtained from pressurized containers or chemical reactions. They are simple to design, manufacture and control, and are lightweight and usually inexpensive. They are scalable to different sizes and resistant to many types of damage. \cite{Shepherd2011,Onal2017}.

\subsubsection{Actuator Shapes}

\hl{linear extension and resulting motion and application}

\hl{torsional extension and resulting motion and application}
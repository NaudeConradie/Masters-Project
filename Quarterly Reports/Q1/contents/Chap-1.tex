\chapter{Introduction}
\label{chp:Intro}


%%%%%%%%%%%%%%%%%%%%%%%%%%%%%%%%%%%%%%%%%%%%%%%%%%%%%%%%%%%%%%%%%%%%%%%
\section{Background}

Manufacturing capabilities have greatly increased over the past decades with the advent of three dimensional (3D) printing and other advanced manufacturing technologies. Development of products and systems are now often limited by design capabilities. Improving and creating new design methods may lead to innovative designs not yet seen before.

The field of soft robotics is particularly suited for creative and novel approaches to the design of components. Soft robotic geometries are often highly complex and free-form. Soft robotic actuators may move in imprecise and non-trivial manners. Materials used in the construction of soft robotic components may behave with non-linear responses. Novel soft robotic actuator designs with new and non-trivial behaviours are actively being created and implemented.

A computationally efficient generative design approach may be used to explore the design space of soft robotics. Generative design allows for the exploration 
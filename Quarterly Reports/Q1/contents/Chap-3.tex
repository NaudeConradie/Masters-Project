\chapter{Material Testing}
\label{chp:MT}

%%%%%%%%%%%%%%%%%%%%%%%%%%%%%%%%%%%%%%%%%%%%%%%%%%%%%%%%%%%%%%%%%%%%%%%
\section{Motivation}

\subsection{Material}

Mold Star 15 SLOW is selected as the main material to be digitally modelled for the purposes of the project. Mold Star 15 is selected because of its availability and properties. Mold Star 15 is deliverable to the premises where testing is to be done and available from a registered supplier. Additional properties are listed in~\ref{tab:gmp} below.

\begin{table}[h]
	\centering
	\caption{Given Material Properties}
	\label{tab:gmp}
	\resizebox{\textwidth}{!}{%
		\begin{tabular}{@{}ccccc@{}}
		\toprule
		\textbf{Material} & \textbf{Cost/kg} & \textbf{Pot life (min)} & \textbf{Cure time (hr)} & \textbf{Tensile strength (MPa)} \\ \midrule
		Mold Star 15 SLOW & R332.00 & 50 & 4 & 2.7579 \\ \bottomrule
		\end{tabular}%
	}
\end{table}


\subsection{Testing}

Standardised testing of the materials to be modelled digitally is done in order to obtain accurate physical properties. The necessary properties to model the material chosen 
\chapter{Material Testing}
\label{chp:MT}

%%%%%%%%%%%%%%%%%%%%%%%%%%%%%%%%%%%%%%%%%%%%%%%%%%%%%%%%%%%%%%%%%%%%%%%
\section{Motivation}

\subsection{Material}

Mold Star 15 SLOW is selected as the main material to be digitally modelled for the purposes of the project. Mold Star 15 is selected because of its availability and properties. Mold Star 15 is deliverable to the premises where testing is to be done and available from a registered supplier. Additional properties are listed in~\ref{tab:gmp} below.

\begin{table}[h]
	\centering
	\caption{Given Material Properties \cite{MoldStar}}
	\label{tab:gmp}
	\resizebox{\textwidth}{!}{%
		\begin{tabular}{@{}ccccc@{}}
		\toprule
		\textbf{Material} & \textbf{Cost/kg} & \textbf{Pot life (min)} & \textbf{Cure time (hr)} & \textbf{Tensile strength (MPa)} \\ \midrule
		Mold Star 15 SLOW & R332.00 & 50 & 4 & 2.7579 \\ \bottomrule
		\end{tabular}%
	}
\end{table}

Mold Star 15's long pot life allows for adequate time to prepare specimens thoroughly. The two components of the material need to be mixed according to a 1:1 ratio and stirred until completely mixed. The material then needs to be degassed, poured into the moulds, degassed again, and levelled.

Mold Star 15 has a hyper-elastic non-linear response. It is suitable for inflation while being capable of supporting itself at the relevant scale of construction.

\subsection{Testing}

The necessary properties to correctly model Mold Star 15 are not available from the supplier. Testing is done according to ISO 37 and ISO 7743 standards for tensile and compression testing respectively to obtain data to construct an accurate Ogden model for the material \cite{ISO37,ISO7743}.

\section{Testing Procedure}

\chapter{Material Testing}
\label{chp:MT}

%%%%%%%%%%%%%%%%%%%%%%%%%%%%%%%%%%%%%%%%%%%%%%%%%%%%%%%%%%%%%%%%%%%%%%%
\section{Motivation}

\subsection{Material}

Mold Star 15 SLOW is selected as the material to be utilised in this project. Mold Star 15 is selected because of its availability and properties. Mold Star 15 is deliverable to the premises where testing is to be done and available from a registered supplier. Additional properties are listed in Table~\ref{tab:ms15gp} below.

\begin{table}[H]
	\centering
	\caption{Mold Star 15 Given Properties \cite{MoldStar}}
	\label{tab:ms15gp}
	\resizebox{\textwidth}{!}{%
		\begin{tabular}{@{}ccccc@{}}
		\toprule
		\textbf{Material} & \textbf{Cost/kg} & \textbf{Pot life (min)} & \textbf{Cure time (hr)} & \textbf{Tensile strength (MPa)} \\ \midrule
		Mold Star 15 SLOW & R332.00 & 50 & 4 & 2.7579 \\ \bottomrule
		\end{tabular}%
	}
\end{table}

Mold Star 15's long pot life allows for adequate time to prepare specimens thoroughly. 

\subsection{Testing}

The necessary material properties to accurately model Mold Star 15 are not available from the supplier. Testing is done according to ISO 37 and ISO 7743 standards for tensile and compression testing respectively to obtain data to construct an accurate Ogden model for the material \cite{ISO37,ISO7743}.

\section{Testing Procedure}

\subsection{Specimen Preparation}

Mold Star 15 requires a specific preparation process to ensure the consistency of the material properties and the usability of the product.

A clean workspace and tools are required for the preparation of the material. Dirt and other materials may cause impurities in the material. Impurities will result in deviations from standard material behaviour. The workspace and tools are wiped down thoroughly with tissue paper. Moulds are sprayed with a non-stick spray. Nitrile gloves are worn. Latex gloves are not appropriate as they may react with the material.

Mold Star 15 consists of two components, Mold Star 15 A and Mold Star 15 B. The two components need to be mixed according to a 1:1 ratio. A clean, empty mixing bowl is placed on an electronic scale. The scale is zeroed. An appropriate mass of Mold Star 15 A is poured into the mixing bowl. The mass added is noted. The scale is zeroed again. A matching mass of Mold Star 15 B is poured into the mixing bowl. Once both materials have been added together, the pot life of 45 minutes starts.

The materials are mixed using a wooden tongue depressor. The materials are mixed until the colour is consistent throughout and no streaks are visible.

The mixed material is degassed until 1 minute after no more bubbles appear. The degassed material is poured into selected moulds. The material is degassed until 1 minute after no more bubbles appear again. Excess material is carefully and slowly removed from the moulds using scrapers or other flat materials. This should be completed before the pot life of 45 minutes has ended.

The material takes 4 hours to set. The material may be baked as well for shorter periods of time. Baking was not used in order to keep material properties consistent and because time constraints were not relevant to material preparation. Set materials may be removed from moulds.

\subsection{Specimen Testing}

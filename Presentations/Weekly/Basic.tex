\documentclass[serif, pdf]{beamer}

\usetheme{Warsaw}

\usepackage{xcolor}
\usepackage{tikz}
\usepackage{multimedia}
%\usepackage{media9}
%\usepackage{movie15}
\usepackage{lmodern}
\usepackage{scrextend}
\usepackage{subcaption}
\usepackage{makecell}				%To keep spacing of text in tables
\setcellgapes{4pt}					%parameter for the spacing
\usepackage{caption}
\captionsetup[figure]{labelformat=empty}% redefines the caption setup of the figures environment in the beamer class.
\setbeamerfont{caption}{size=\scriptsize}


\usecolortheme{beaver}

\title[SEMC]{}
\date{August 2019}
\author[Naud\e' Conradie]{Naud\e' Conrade\\{\small Co-author: Dr M Venter}}
\institute[]{Department of Mechanical Engineering, Stellenbosch University}


\definecolor{colour1}{RGB}{96, 34, 59}
\definecolor{colour2}{RGB}{140, 151, 154}
\setbeamercolor{structure}{fg=colour1,bg=colour2}
\setbeamercolor{title}{fg=white,bg=colour2}
\setbeamercolor{author in head/foot}{fg=colour1}

\setbeamertemplate{itemize item}{\color{black}$\bullet$}
\setbeamertemplate{itemize subitem}{\color{black}$-$}
\setbeamertemplate{caption}{\raggedright\insertcaption\par}

\expandafter\def\expandafter\insertshorttitle\expandafter{%
  \insertshorttitle\hfill%
  \hspace{30mm}\insertframenumber\,/\,\inserttotalframenumber}

\beamertemplatenavigationsymbolsempty

\begin{document}

%----------------------Title page----------------------%

\begin{frame}
  \begin{center}
    \vspace{0.1cm}
    \includegraphics[scale=0.25]{USlogo.pdf}
  \end{center}
  \titlepage
\end{frame}

%--------------Introduction---------------%

\changefontsizes{13pt}
\begin{frame}
\frametitle{Aim}
\begin{itemize}
\uncover<1->{\item Create numerical model to model stiffness and inflated shape of simple bag}
\uncover<2->{\item Increase complexity of model}
\uncover<3->{\item Use model for shape optimisation for inflatable wing}
\end{itemize}
\begin{figure}
\uncover<3->{\includegraphics[height=1.7cm]{Trailing_Edge_1.jpg}}
\end{figure}
\end{frame}

%--------------Inflatable Wings---------------%

\changefontsizes{13pt}
\begin{frame}
\frametitle{Inflatable Wings - Planes}

\begin{figure}
\includegraphics[height=5cm]{Inflatoplane.jpg}
\end{figure}
\end{frame}

%---------------Methodology------------------------------%

\begin{frame}
\frametitle{Methodology}
\begin{itemize}
\item<1-> Measure stiffness through 4 point bending tests
\item<2-> Capture physical model geometry with 3D scanner
\item<3-> Construct finite element models of single cavity bags
\begin{itemize}
\item<3-> LS-DYNA's explicit solver
\end{itemize}
\end{itemize}
%\begin{figure}
%\includegraphics[height=2cm]{4-Point_Bending_Test}<1->
%\end{figure}
\end{frame}

%-------------Material-------------------------------%

%\begin{frame}
%\frametitle{Material-Polyethylene}
%\begin{itemize}
%\item<1-> Readily available
%\item<2-> 85$\mu$ thickness
%\item<3-> Uni-axial tensile testing
%\end{itemize}
%\begin{figure}
%\includegraphics<4->[scale=0.4]{Tensile_Test_Data}
%\end{figure}
%\end{frame}

%------------------Finite Element Model-----------%

\begin{frame}
\frametitle{FE Model - 4 Point Bending}
\begin{minipage}{0.60\textwidth}
\begin{itemize}
\item<1-> Polyethylene - Linear elastic membrane
\item<2-> Airbag model for inflation 
\item<3-> Two boundary condition types used
\item<4-> Dynamic relaxation
\item<5-> Mass scaling
\end{itemize}
\end{minipage}
\begin{minipage}{0.35\textwidth}
\begin{figure}
\includegraphics[height=3.3cm]{Tensile_Testing.pdf}<1->
\end{figure}
\end{minipage}


\end{frame}

%------------------4 Point Bending Tests Physical------------%

\begin{frame}
\frametitle{4 Point Bending Tests}
\begin{itemize}
\item<1-> Measure force vs. displacement
\item<2-> Diameter 50, 75 and 100~mm tubes
\item<3-> 3 different pressures tested per tube
\end{itemize}
\begin{figure}
\includegraphics[scale=0.4]{4_Point.jpg}
\end{figure}
\end{frame}

%------------------4 Point Bending Tests Simulation------------%

\begin{frame}
\frametitle{4 Point Bending Tests}
%\begin{figure}
%\includegraphics[scale=0.5]{50_All.png}
%\caption{Force vs. displacement curve for all 3 pressures of 50~mm bag}
%\end{figure}

\begin{figure}[h!]
\begin{subfigure}{.45\textwidth}
  \centering
  \includegraphics[width=1\linewidth]{All_5_75.pdf}
  \caption{\scriptsize{Bag size comparison at 5.75~kPa.}}
  \label{fig:Point_Cloud_75_4-Point_yz}
\end{subfigure}
\begin{subfigure}{.46\textwidth}
  \centering
  \includegraphics[width=1\linewidth]{50_All.pdf}
  \caption{\scriptsize{Pressure comparison for 50~mm bag.}}
  \label{fig:Point_Cloud_75_4-Point_xz}
\end{subfigure}%
%\caption{Symmetry plane plots comparing the shape of the numerical and physical model of a 75~mm inflated bag under 4-point bending.}
\label{fig:Point_Cloud_75_Load}
\end{figure}
\end{frame}



%------------------4 Point Bending Tests Simulation cont... ------------%

\begin{frame}
\frametitle{4 Point Bending Tests}
At displacement of 40~mm:
\begin{itemize}
\item<1-> At 5.75~kPa a 100$\%$ increase in bag size yields a 375$\%$ increase in stiffness
\item<2-> For the 50~mm bag a 160$\%$ increase in pressure the stiffness increases by 46$\%$
\item<3-> Increase on bag size has a larger influence on the stiffness than the pressure
\end{itemize}
\end{frame}

%------------------4 Point Bending Tests Simulation cont... ------------%

%\begin{frame}
%\frametitle{4 Point Bending Tests - $\O$50 tube}
%\begin{figure}
%\includegraphics[scale=0.5]{50_9.png}
%\caption{Force vs. displacement curve for tube at 9~kPa}
%\end{figure}
%\end{frame}

%------------------4 Point Bending Tests Simulation cont... ------------%

%\begin{frame}
%\frametitle{4 Point Bending Tests - $\O$50 tube}
%\begin{figure}
%\includegraphics[scale=0.5]{50_15.png}
%\caption{Force vs. displacement curve for tube at 15~kPa}
%\end{figure}
%\end{frame}

%------------------4 Point Bending Tests Simulation cont... ------------%

%\begin{frame}
%\frametitle{4 Point Bending Tests - 50~mm Bag}
%\begin{figure}[h!]
%\begin{subfigure}{.5\textwidth}
%  \centering
%  \includegraphics[width=0.8\linewidth]{50_5_75.pdf}
%  \caption{\scriptsize{50~mm bag at 5.75~kPa.}}
%  \label{fig:50_5_75_Numerical}
%\end{subfigure}%
%\begin{subfigure}{.5\textwidth}
%  \centering
%  \includegraphics[width=0.8\linewidth]{50_9.pdf}
%  \caption{\scriptsize{50~mm bag at 9~kPa.}}
%  \label{fig:50_9_Numerical}
%\end{subfigure}
%\begin{subfigure}{\textwidth}
%  \centering
%  \includegraphics[width=.4\linewidth]{50_15.pdf}
%  \caption{\scriptsize{50~mm bag at 15~kPa.}}
%  \label{fig:50_15_Numerical}
%\end{subfigure}
%%\caption{The numerical force vs. displacement values for the 50~mm bags plotted with the data from the physical tests.}
%\label{fig:50_Numerical}
%\end{figure}
%\end{frame}

%------------------4 Point Bending Tests Simulation cont... ------------%

\begin{frame}
\frametitle{4 Point Bending Tests - 75~mm Bag}
\begin{figure}[h!]
\begin{subfigure}{.5\textwidth}
  \centering
  \includegraphics[width=0.8\linewidth]{75_5_75.pdf}
  \caption{\scriptsize{75~mm bag at 5.75~kPa.}}
  \label{fig:50_5_75_Numerical}
\end{subfigure}%
\begin{subfigure}{.5\textwidth}
  \centering
  \includegraphics[width=0.8\linewidth]{75_6_75.pdf}
  \caption{\scriptsize{75~mm bag at 6.75~kPa.}}
  \label{fig:50_9_Numerical}
\end{subfigure}
\begin{subfigure}{\textwidth}
  \centering
  \includegraphics[width=.4\linewidth]{75_9.pdf}
  \caption{\scriptsize{75~mm bag at 9~kPa.}}
  \label{fig:50_15_Numerical}
\end{subfigure}
%\caption{The numerical force vs. displacement values for the 50~mm bags plotted with the data from the physical tests.}
\label{fig:50_Numerical}
\end{figure}
\end{frame}

%------------------4 Point Bending Tests Simulation cont... ------------%

%\begin{frame}
%\frametitle{4 Point Bending Tests - 100~mm Bag}
%\begin{figure}[h!]
%\begin{subfigure}{.5\textwidth}
%  \centering
%  \includegraphics[width=0.8\linewidth]{100_4_75.pdf}
%  \caption{\scriptsize{100~mm bag at 4.75~kPa.}}
%  \label{fig:50_5_75_Numerical}
%\end{subfigure}%
%\begin{subfigure}{.5\textwidth}
%  \centering
%  \includegraphics[width=0.8\linewidth]{100_5_75.pdf}
%  \caption{\scriptsize{100~mm bag at 5.75~kPa.}}
%  \label{fig:50_9_Numerical}
%\end{subfigure}
%\begin{subfigure}{\textwidth}
%  \centering
%  \includegraphics[width=.4\linewidth]{100_6_75.pdf}
%  \caption{\scriptsize{100~mm bag at 6.75~kPa.}}
%  \label{fig:50_15_Numerical}
%\end{subfigure}
%%\caption{The numerical force vs. displacement values for the 50~mm bags plotted with the data from the physical tests.}
%\label{fig:50_Numerical}
%\end{figure}
%\end{frame}

%------------------4 Point Bending Tests Simulation cont... ------------%

%\begin{frame}
%\frametitle{4 Point Bending Tests Validation}
%\begin{table}[h!]
%\makegapedcells
%\begin{center}
%\caption{A statistical measure of the deviations in resulting force between the numerical and physical model for all 3 bags under a 4-point bending load.}
%\label{tab:50_4-point_quant}
%\centering
% \begin{tabular}{l l c c c p{3cm}} 
% \hline
% Bag & Pressure & Max $\%$ error & Mean $\%$ error & $\sigma$ & Displacement at max $\%$ error \\
% \hline \hline
% 50~mm & 5.75~kPa & 11.76 & 6.54 & 5.08 & 30~mm\\ 
% & 9~kPa & 7.88 & 4.59 & 3.00 & 30~mm\\
% & 15~kPa & 18.52 & 10.87 & 7.32 & 10~mm\\
% 75~mm & 5.75~kPa & 18.23 & 10.89 & 6.53 & 20~mm\\ 
% & 6.75~kPa & 17.00 & 11.00 & 4.97 & 40~mm\\
% & 9~kPa & 13.04 & 8.10 & 4.25 & 40~mm\\
% 100~mm & 4.75~kPa & 51 & 21.18 & 21.21 & 5~mm\\ 
% & 5.75~kPa & 57.14 & 22.04 & 22.90 & 5~mm\\
% & 6.75~kPa & 47.43 & 20.35 & 18.89 & 5~mm\\
% \hline
%\end{tabular}
%\end{center}
%\end{table}
%\end{frame}

%-------------------------Shape Validation--------------------------------%

\begin{frame}
\frametitle{Shape Validation}
\begin{itemize}
\item<1-> Point clouds of bags both under load and no load compared
\item<2-> Metamodel based optimisation used to minimize Hausdorff distance
\item<3-> Rigid body transformations used to rotate physical model 
\end{itemize}
\uncover<2->{
\begin{align*}
\text{Minimise}  \, f(\textbf{x}) & \\
\text{where} \, f(\textbf{x}) & = \text{Hausdorff Distance}  \\
\text{and} \, \, \textbf{x} & = [T_x ,T_y ,T_z ,R_x ,R_y ,R_z] 
\end{align*}}

%  \begin{align}
%     \tilde{A} &= {\Omega}^{2} \, A,
%     &\tilde{B} &= {\Omega}^{- 2} \, B, \nonumber \\[12pt]
%     \tilde{\Phi} &= {\Omega}^{w} \, \Phi,
%     &\tilde{\Psi} &= {\Omega}^{w} \, \Psi,
%   \end{align}

\end{frame}

%-------------------------Shape Validation cont...------------------------%
\begin{frame}
\frametitle{Hausdorff Distance}
\begin{center}
\includegraphics[scale=0.7]{Hausdorff}
\end{center}
\end{frame}

%-------------------------Shape Validation cont...------------------------%

\begin{frame}
\frametitle{Shape Validation - 75~mm Bag Under No Load}
\begin{figure}[h!]
\begin{subfigure}{.5\textwidth}
  \centering
  \includegraphics[width=0.94\linewidth]{Tube_75_xz.pdf}
  \label{fig:Point_Cloud_75_No-Load_xz}
\end{subfigure}%
\begin{subfigure}{.5\textwidth}
  \centering
  \includegraphics[width=0.94\linewidth]{Tube_75_yz.pdf}
  \label{fig:Point_Cloud_75_No-Load_yz}
\end{subfigure}
\begin{subfigure}{\textwidth}
  \centering
  \includegraphics[width=.47\linewidth]{Tube_75_xy.pdf}
  \label{fig:Point_Cloud_75_No-Load_xy}
\end{subfigure}
%\caption{Symmetry plane plots comparing the shape of the numerical and physical model of a 75~mm inflated bag under no no load.}
%\label{fig:Point_Cloud_75_No-Load}
\end{figure}
\end{frame}

%-------------------------Shape Validation cont...------------------------%

\begin{frame}
\frametitle{Shape Validation - 75~mm Bag Under Load}
\begin{figure}[h!]
\begin{subfigure}{.5\textwidth}
  \centering
  \includegraphics[width=0.94\linewidth]{4-point_75_xz.pdf}
  \label{fig:Point_Cloud_75_4-Point_xz}
\end{subfigure}%
\begin{subfigure}{.5\textwidth}
  \centering
  \includegraphics[width=0.94\linewidth]{4-point_75_yz.pdf}
%  \caption{\scriptsize{Plot of points on the $yz$ plane for the 75~mm bag.}}
  \label{fig:Point_Cloud_75_4-Point_yz}
\end{subfigure}
%\caption{Symmetry plane plots comparing the shape of the numerical and physical model of a 75~mm inflated bag under 4-point bending.}
%\label{fig:Point_Cloud_75_Load}
\end{figure}
\begin{itemize}
\item Larger displacement from physical bag
\item Best fit on $zy$ plane
\end{itemize}
\end{frame}



%-------------------------Shape Optimisation cont...------------------------%

\begin{frame}
\frametitle{Shape Optimisation}
\begin{center}
\begin{itemize}
\item<1-> Parametrise movement in the $x$ and $z$ directions 
\item<2-> Minimise RMSE between NACA 0030 aerofoil
\end{itemize}
\end{center}

\begin{figure}
\includegraphics[scale=0.4]{Initial_Opt_State}<1->
\end{figure}
\end{frame}

%-------------------------Shape Optimisation cont...------------------------%

\begin{frame}
\frametitle{Shape Optimisation}
\begin{figure}[h!]
\begin{subfigure}{.5\textwidth}
  \centering
  \uncover<1->{\scalebox{-1}[1]{\includegraphics[height=2.2cm]{3_Bag_Wing}}}
  \label{fig:50_5_75_Numerical}
\end{subfigure}%
\begin{subfigure}{.5\textwidth}
  \centering
  \uncover<2->{\scalebox{-1}[1]{\includegraphics[height=2.2cm]{8_Bag_Wing}}}
  \label{fig:50_9_Numerical}
\end{subfigure}
\begin{subfigure}{\textwidth}
  \centering
  \uncover<3->{\scalebox{-1}[1]{\includegraphics[height=2.2cm]{15_Bag_Wing}}}
  \label{fig:50_15_Numerical}
\end{subfigure}
%\caption{The numerical force vs. displacement values for the 50~mm bags plotted with the data from the physical tests.}
\label{fig:50_Numerical}
\end{figure}
\end{frame}

%-------------------------Shape Optimisation cont...------------------------%

\begin{frame}
\frametitle{Shape Optimisation - 3 Bag Aerofoil}
\begin{figure}[h!]
  \includegraphics[width=1\linewidth]{Aerofoil_3}
\end{figure}
\end{frame}

%-------------------------Shape Optimisation cont...------------------------%

\begin{frame}
\frametitle{Shape Optimisation - 8 Bag Aerofoil}
\begin{figure}[h!]
  \includegraphics[width=1\linewidth]{Aerofoil_8}
\end{figure}
\end{frame}

%-------------------------Shape Optimisation cont...------------------------%

\begin{frame}
\frametitle{Shape Optimisation - 15 Bag Aerofoil}
\begin{figure}[h!]
  \includegraphics[width=1\linewidth]{Aerofoil_15}
\end{figure}
\end{frame}

%-------------------------Shape Optimisation cont...------------------------%

\begin{frame}
\frametitle{Shape Optimisation -  Bag Comparison}
\begin{table}[H]
\makegapedcells
\begin{center}
 \begin{tabular}{lccc} 
 \hline
  & 3 Bag & 8 Bag & 15 Bag \\
 \hline \hline
 RMSE & 0.02606 & 0.00844 & 0.00453\\ 
 $R^2$ & 0.557 & 0.961 & 0.990 \\
 \hline
\end{tabular}
\end{center}
\end{table}
\end{frame}

%------------------------Physical Model--------------------------------%

\begin{frame}
\frametitle{Physical Model - 8 Bag Aerofoil}
\begin{figure}[h!]
  \scalebox{-1}[1]{\includegraphics[height=6.5cm]{Physical_8_Bag_Aerofoil}}
\end{figure}
\end{frame}

%------------------------Physical Comparision--------------------------------%

\begin{frame}
\frametitle{Physical Comparison - 8 Bag Aerofoil}
\begin{figure}[h!]
  \includegraphics[width=0.9\linewidth]{Physical_Comparison_8.pdf}
\end{figure}
\begin{center}
$R^2 = 0.941$
\end{center}
\end{frame}

%------------------------Physical Comparision--------------------------------%

\begin{frame}
\frametitle{Concluding Remarks}
\begin{itemize}
\item Able to achieve high $R^2$ fit from between physical profile and target profile.
\item More accurate material model could help achieve better fit. 
\end{itemize}
\end{frame}


%------------------------Questions--------------------------------%

\begin{frame}
\begin{center}
\huge Questions?
\end{center}
\end{frame}




\end{document}

%\begin{figure}
%\includegraphics[width=0.25\textwidth]{Pictures/Plain_Weave.jpg}
%\caption{Plain weave architecture}
%\end{figure}



%\begin{itemize}
%\setlength\itemsep{1em}
%\item<1->{Plain weave architecture.}
%\item<2->{Orthogonal material directions.}
%\item<3->{Behaves unlike conventional metallic materials.}
%\item<4->{Coupled response due to crimp interchange.}
%\end{itemize}


%\begin{block}{\textcolor{black}{}}
%XXX
%\end{block}

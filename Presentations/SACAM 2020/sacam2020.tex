\documentclass[a4paper]{sacam2020}

\usepackage{parskip}
\usepackage{pslatex}
\usepackage{graphicx}
\usepackage{amsmath}
\usepackage{amsfonts}
\usepackage{amssymb}
\usepackage{hyperref}

\setlength{\parskip}{5pt}

\title{The virtual evolution of 2D soft robots}

\author{Naude T. Conradie$^{*}$ and Martin P. Venter$^{\dag}$}

\address{$^{*}$ Stellenbosch University\\
Huis Russel Botman, Marais Road, Stellenbosch, 7600\\
\email{19673418@sun.ac.za}
\and
$^{\dag}$ Stellenbosch University\\
Faculty Of Engineering, Stellenbosch University, Private Bag X1, Matieland, 7602\\
\email{mpventer@sun.ac.za}}

%\keywords{Instructions, Minisymposium, Computational Mechanics, Fluid Dynamics}

\begin{document}

\thispagestyle{empty}

\section*{ABSTRACT}

Soft robots are difficult to design, and a large area of the design space remains unexplored. Using genetic algorithms is an effective method to explore the design space \cite{Sims1994a}. However, the simulation and prediction of the behaviour of soft bodies is computationally expensive due to their physical properties. An efficient method of representing these bodies is desirable \cite{Hiller2010}. This would allow for faster and more effective design processes to be widely implemented.

To prove the merits of virtually evolving soft robotic bodies, two-dimensional bodies are evolved using sequential iterations of numerical optimization. Bodies are composed of unit cells with defined behaviours and responses to an applied internal pressure. Complete soft bodies are built using generalized recursive encodings such as Lindenmayer systems (L-systems) and CPPN-NEAT. These bodies behave according to real-world physics and are modelled representatively. Material models appropriate for non-linear hyper-elastic FEM will be used.

From the representative encodings, the soft bodies are built first at a cellular level using L-systems and the unit cells, and then wholly using CPPN-NEAT. The bodies are obtained with greatly improved computing times regarding both their evolution and modelling. The methodology is easily replicable and adaptable. Physical replicas of unit cells and some well-performing whole bodies are produced as a proof of concept.

Generalized recursive encodings allow for much faster and more efficient evolution of soft bodies, due to their inherently compactible nature. They may be used going forward to obtain soft robotic models for a wide range of applications.

\bibliography{sacam2020}
\bibliographystyle{ieeetr}

\end{document}


